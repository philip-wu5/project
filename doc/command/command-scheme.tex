\documentclass[11pt]{article}


\usepackage{fancyhdr}

%\usepackage{epsfig}
\usepackage{amssymb}
\usepackage{amstext}
\usepackage{amsmath}
\usepackage{textcomp}
\usepackage{stmaryrd}
\usepackage{graphicx}
\usepackage{tabls}
\usepackage[mathcal]{euscript}
\usepackage{moreverb}
\usepackage{setspace}

\usepackage{courier}
\usepackage{url}
%\usepackage{algorithmicx}
%\usepackage[ruled]{algorithm}
%\usepackage[noset]{algpseudocode}
%\usepackage[noset]{algpascal}
%\usepackage[noset]{algc}
\usepackage{fancyvrb}



\def \tit{\textit}
\def \tsf{\textsf}
\def \ttt{\texttt}
\def \tbf{\textbf}
\def \vb{\verb}


\usepackage{draftwatermark}

\SetWatermarkText{DRAFT v.0.1}
\SetWatermarkLightness{0.95}
\SetWatermarkScale{3}

\begin{document}



%\title{A Very Simple \LaTeXe{} Template}
\title{\ttt{vcluster} Command Structure}
\author{
        Seo-Young Noh, Dada Huang\thanks{Additional authors will be listed depending on their contributions.} \\
        $ $\\
        National Institute of Supercomputing \& Networking\\
        Korea Institute of Science and Technology Information\\
        Youseong, Daejeon, 305-806, Korea\\
        {rsyoung, huang\_dada}@kisti.re.kr
%            \and
%        Yossi Gil\\
%        Department of Computer Science\\
%        Technion---Israel Institute of Technology\\
%        Technion City, Haifa 32000, \underline{Israel}
}
\date{\today}

%\date{}


\maketitle

\begin{abstract}
This report provides the command scheme for \ttt{vcluster}.
\end{abstract}


%---------------------------------------------------------------------------
\section{TODO List}

\begin{itemize}
 \item Need to articulate this article
 \item Need to add outputs of commands
\end{itemize}




\section{Overall View of Commands}

A \vb+vcluster+ command consists of \tbf{Command Category} and \tbf{Real Command}. Command category indicates where a real command is belonging to. For example, we have to use \vb+plugman+ as a command category when handling plugin related works. Following shows command categories implemented (or to be implemented) in \vb+vcluster+.

\begin{Verbatim}[fontfamily=courier, fontsize = \small, obeytabs
=true, tabsize=4, frame=lines]

      cloudman          command
      plugman           command
      vmman             command
      [vclman]          command
      ...               ...
  
\end{Verbatim}

There is a special command category called \vb+vclman+ which can be omitted. Types of commands belonging to this category are including configurations, start and stop of \vb+vcluster+.


\section{Commands of \ttt{plugman} Category}

All commands after \vb+plugman+ are plugin related ones. Such commands are including load, unload, list of plugins. There are two types of plugins which are batch plugin and cloud plugin, respectivley. We will discuss plugin related commands in the following subsections.

Like a general Linux command, \vb+-h+ or \vb+--help+ option shows the usages of \vb+plugman+ and commands.

\begin{Verbatim}[fontfamily=courier, fontsize = \small, obeytabs
=true, tabsize=4, frame=lines]

  plugman 
    -h, --help     list up all options and usages
  
\end{Verbatim}


\subsection{\ttt{load}}
This command loads a plugin or a bunch of plugins. Since one batch system plugin is only allowed at the same time, the structure of \vb+load+ command depends on the type of plugin.  

\begin{Verbatim}[fontfamily=courier, fontsize = \small, obeytabs
=true, tabsize=4, frame=lines]

plugman 
    load
      -c PLUGIN... | --type=cloud PLUGIN... 
      -b BATCH_PLUGIN | --type=batch PLUGIN
      
\end{Verbatim}


The options \vb+-c+ and \vb+--type=cloud+ are identical. These options are saying that we are about to load cloud type plugin(s). Like the cloud type options, we can use \vb+-b+ and \vb+--type=batch+ options for a batch type plugin. Pleas note that unlike a cloud type plugin, only one batch plugin should be provided.

Examples for this command are as below:


\begin{Verbatim}[fontfamily=courier, fontsize = \small, obeytabs
=true, tabsize=4, frame=lines]

vcluster> plugman load -c plugin-1
vcluster> plugman load --type=cloud plugin-1

vcluster> plugman load -c plugin-1 plugin-2 plugin-3
vcluster> plugman load --type=cloud plugin-1 plugin-2 plugin-3

vcluster> plugman load -b plugin-1
vcluster> plugman load --type=batch plugin-1

vcluster> plugman load -b plugin-1 plugin-2 plugin-3
vcluster> plugman load --type=batch plugin-1 plugin-2 plugin-3

\end{Verbatim}



\subsection{\ttt{unload}}
This command unloads a plugin or a bunch of plugins. This command unlike \vb+load+ command does not indicate the type of plugin(s) to be unloaded.

\begin{Verbatim}[fontfamily=courier, fontsize = \small, obeytabs
=true, tabsize=4, frame=lines]

plugman 
    unload PLUGIN...
      
\end{Verbatim}

Below shows an example of this command.

\begin{Verbatim}[fontfamily=courier, fontsize = \small, obeytabs
=true, tabsize=4, frame=lines]

vcluster> plugman unload plugin-1
vcluster> plugman unload plugin-1 plugin-2

\end{Verbatim}




\subsection{\ttt{list}}
This command lists up all designated plugins. Since \vb+vcluster+ does not have \vb+register+ command, it retrieves all plugins under a specifided plugin directory. When listing up plugins currently being used, option \vb+-l+ or \vb+--loaded+ can be used at the end of command.

\begin{Verbatim}[fontfamily=courier, fontsize = \small, obeytabs
=true, tabsize=4, frame=lines]

plugman 
    list
      -c | --type=cloud 
      -b | --type=batch
      -l | --loaded
      
\end{Verbatim}

You may combine \vb+-c+ and \vb+-l+ options to show up all loaded cloud plugins. The following shows examples of this command.

\begin{Verbatim}[fontfamily=courier, fontsize = \small, obeytabs
=true, tabsize=4, frame=lines]

vcluster> plugman list -c
vcluster> plugman list --type=cloud

vcluster> plugman list -b
vcluster> plugman list --type=batch

vcluster> plugman list -l
vcluster> plugman list --loaded

vcluster> plugman list -l -c
vcluster> plugman list -l --type=cloud
vcluster> plugman list --loaded -c

vcluster> plugman list -l -b
vcluster> plugman list -l --type=batch
vcluster> plugman list --loaded -b

\end{Verbatim}

\textcolor{red}{TODO: when listing up all plugins under a directory, the output should exlicitly mention that the outputs are coming from a directory, not from memory.}


\subsection{\ttt{info}}
This command prints the information about a plugin. It will be used to retrieve detail information about the plugin. \textcolor{red}{TODO: plugin interface needs to provide this feature. We may need to introduce a structure containing required fields for this command.}

\begin{Verbatim}[fontfamily=courier, fontsize = \small, obeytabs
=true, tabsize=4, frame=lines]

plugman 
    info PLUGIN
      
\end{Verbatim}

Below shows an example of this command.

\begin{Verbatim}[fontfamily=courier, fontsize = \small, obeytabs
=true, tabsize=4, frame=lines]

vcluster> plugman info plugin-1
      
\end{Verbatim}



\end{document}


